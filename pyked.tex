% !TEX option = -shell-escape
%%%%%%%%%%%%%%%%%%%%%%%%%%%%%%%%%%%%%%%%%%%%%%%%%%%%%%%%%%%%%%%%%%%%%%%
% This work is licensed under the Creative Commons Attribution 4.0
% International License. To view a copy of this license, visit
% http://creativecommons.org/licenses/by/4.0/.
%%%%%%%%%%%%%%%%%%%%%%%%%%%%%%%%%%%%%%%%%%%%%%%%%%%%%%%%%%%%%%%%%%%%%%%
\documentclass[12pt]{ussci}

\usepackage{todonotes}
\newcommand\ck{ChemKED}
\usepackage{enumitem}
\setlist{noitemsep}
%======================================================================
\addbibresource{pyked.bib}
%======================================================================
\newcommand\papertopic{Reaction Kinetics}
%======================================================================

\title{ ChemKED: a human- and machine-readable data standard for chemical kinetics experiments }

\author[1*]{Bryan W. Weber}
\author[2]{Kyle E. Niemeyer}

\affil[1]{Department of Mechanical Engineering, University of Connecticut, Storrs, CT, USA}
\affil[2]{Mechanical, Industrial, and Manfacturing Engineering, Oregon State University, Corvalis, OR, USA}
\affil[*]{Corresponding author: \email{bryan.weber@uconn.edu}}

\begin{document}
\maketitle

%====================================================================
\begin{abstract} % not to exceed 200 words
Fundamental experimental measurements of quantities including ignition delay
times, laminar flame speeds, and species profiles serve important roles in
understanding fuel chemistry and validating chemical kinetic models. However,
despite both the importance and abundance of such information in the literature,
the community lacks a widely adopted standard format for this data. This impedes
both sharing and wide use by the community. In this talk, we will introduce a
new Chemical Kinetics Experimental Data format, ChemKED, and the related
Python-based package for creating and validating ChemKED-formatted files called
PyKED. We will also review past and related efforts, and motivate the need for a
new solution. ChemKED currently supports the representation of autoignition
delay time and laminar flame speed measurements. ChemKED-formatted files contain
all of the information needed to simulate experimental data points, including
uncertainty. ChemKED is based on the YAML data serialization language, and is
intended as a human- and machine-readable standard for easy creation and
automated use. Development of ChemKED and PyKED occurs openly on GitHub under
the BSD 3-clause license, and contributions from the community are welcome.
Plans for future development include support for experimental data from jet
stirred reactor, extinction, and speciation measurements.
\end{abstract}

% (Provide 2-4 keywords describing your research. Only abbreviations firmly
% established in the field may be used. These keywords will be used for
% sessioning/indexing purposes.)
\begin{keyword}
    chemical kinetics\sep software\sep database
\end{keyword}

%====================================================================
\section{Introduction}
%
\todo[inline]{Kyle to do}


%====================================================================
\section{Related work}
%
\todo[inline]{Kyle to do}


%====================================================================
\section{Overview of format}\label{sec:overview-of-format}

The \ck{} file is a YAML-format file. This format offers the advantage of being
human readable and having parsers in most common programming languages,
including Python, C++, Java, Perl, MATLAB, and many others. The YAML syntax is
quite simple. The basic file structure is made of mappings, delimited by a colon:

\begin{minted}{yaml}
    key: value
    key: 10 # This is because YAML syntax highlighting in Atom is busted
\end{minted}

In addition, sequences can be specified by using a hypen \yaml|-| or square
brackets

\begin{minted}{yaml}
    - [0, 1, 2]
    key:
      - 0
      - 1
      - 2
\end{minted}

The \ck{} format is designed
to include all of the information necessary to simulate a given experiment. Most
of the required information is specified as a \yaml|key| with a \yaml|value|.

In
general, this includes

\todo{Make this smaller}
\begin{itemize}
    \item The type of experiment
    \begin{itemize}
        \item Ignition delay
        \item Flame speed
        \item Speciation
        \item Extinction
    \end{itemize}
    \item The experimental apparatus
    \begin{itemize}
        \item Shock tube
        \item Rapid compression machine
        \item Counterflow flame
        \item Burner-stabilized flame
    \end{itemize}
    \item The thermodynamic conditions of the experiment
    \begin{itemize}
        \item Temperature
        \item Pressure
        \item Composition
    \end{itemize}
    \item Apparatus specific effects
    \begin{itemize}
        \item Non-ideal pressure rise in shock tubes
        \item Post-compression heat loss in rapid compression machines
    \end{itemize}
\end{itemize}


%====================================================================
\section{Use example}
%
\todo[inline]{Bryan/Kyle to do}


%====================================================================
\section{Conclusions and future work}
%
\todo[inline]{both to do}



%====================================================================
\section{Acknowledgements}
This research was funded by \ldots
\todo[inline]{update}


\printbibliography

\end{document}
