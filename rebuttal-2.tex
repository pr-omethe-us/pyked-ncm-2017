% OSUletter_example.tex - an example latex file to illustrate OSUletter.cls
%
% Template by Brian Wood (brian.wood@oregonstate.edu).  Please feel free to send suggestions for changes; this template/cls is not exactly elegantly done!
%
% This template is designed to be easily modified so that it can be used by anyone at OSU.  The Department/School and College are easily modified in this file.  There should be no need to directly alter the .cls file.

\documentclass[11pt]{OSUletter}
\usepackage{tikz}
\usepackage{xcolor}
	% Set up custom OSU orange color
	\definecolor{osuorange}{cmyk}{.1197,.9474,1,0.0309}
\usepackage{lipsum}
\usepackage{fancyhdr}
\usepackage{lastpage}
%
% This section is just a bunch of busywork so that the second and following pages read ``Page X of Y''
\pagestyle{fancy}
\fancyhf{}
\renewcommand{\headrulewidth}{0pt}
\renewcommand{\footrulewidth}{0pt}
\rhead{Page \thepage \hspace{1pt} of \pageref{LastPage}}
%
%
\usepackage[utf8]{inputenc}
\usepackage[T1]{fontenc}
\usepackage[english]{babel}
\usepackage{textcomp}
\usepackage{lmodern}

\usepackage[hyphens]{url}
\usepackage{hyperref}

\usepackage[font={itshape}]{quoting}

\usepackage[binary-units]{siunitx}
\sisetup{group-separator={,},
	detect-all,
	binary-units,
	list-units = single,
	range-units = single,
	range-phrase = --,
	per-mode = symbol-or-fraction,
	separate-uncertainty = true,
	multi-part-units = single,
	list-final-separator = {, and }
%    scientific-notation = fixed
}

\usepackage{fancyvrb,newverbs}
\definecolor{cverbbg}{gray}{0.93}
\colorlet{pyyellow}{yellow!10!white}
\colorlet{yamlblack}{black!10!white}
\newverbcommand{\yabox}
  {\setbox\verbbox\hbox\bgroup}
  {\egroup\colorbox{yamlblack}{\box\verbbox}}
\newverbcommand{\pybox}
  {\setbox\verbbox\hbox\bgroup}
  {\egroup\colorbox{pyyellow}{\box\verbbox}}

\usepackage[normalem]{ulem}
% *** Be very precise and careful about including whitespace and punctuation in your edits ***
% For the version with changes, use these four commands
\newcommand{\addone}[1]{{\sloppy\textcolor{teal}{\uline{#1}}}}  % add
\newcommand{\deleteone}[1]{\sloppy\textcolor{teal}{\sout{#1}}}    % delete
\newcommand{\addtwo}[1]{{\sloppy\textcolor{red}{\uline{#1}}}}  % add
\newcommand{\deletetwo}[1]{\sloppy\textcolor{red}{\sout{#1}}}    % delete
\newcommand{\addnoul}[1]{{\textcolor{teal}{#1}}}     % add with no underline
\newcommand{\deletenoso}[1]{{\textcolor{red}{#1}}}    % delete with no strikeout
%
% The material below is a whole big dang thing whose purpose is just to set up a fixed coordinate system for \tikz so that you can put the Department or School address in the upper right-hand side without it moving all around every time you change something in the page.  I think it works.
% Defining a new coordinate system for the page:
%
% --------------------------
% |(-1,1)    (0,1)    (1,1)|
% |                        |
% |(-1,0)    (0,0)    (1,0)|
% |                        |
% |(-1,-1)   (0,-1)  (1,-1)|
% --------------------------
\makeatletter
\def\parsecomma#1,#2\endparsecomma{\def\page@x{#1}\def\page@y{#2}}
\tikzdeclarecoordinatesystem{page}{
    \parsecomma#1\endparsecomma
    \pgfpointanchor{current page}{north east}
    % Save the upper right corner
    \pgf@xc=\pgf@x%
    \pgf@yc=\pgf@y%
    % save the lower left corner
    \pgfpointanchor{current page}{south west}
    \pgf@xb=\pgf@x%
    \pgf@yb=\pgf@y%
    % Transform to the correct placement
    \pgfmathparse{(\pgf@xc-\pgf@xb)/2.*\page@x+(\pgf@xc+\pgf@xb)/2.}
    \expandafter\pgf@x\expandafter=\pgfmathresult pt
    \pgfmathparse{(\pgf@yc-\pgf@yb)/2.*\page@y+(\pgf@yc+\pgf@yb)/2.}
    \expandafter\pgf@y\expandafter=\pgfmathresult pt
}
\makeatother
%
%
%%%%%%%%%%% Put Personal Information Here %%%%%%%%%%%
%
\def\name{Kyle Niemeyer, PhD\\
Assistant Professor\\
School of Mechanical, Industrial, \& Manufacturing Engineering
}
%
% List your degree(s), licences, etc. here if you like.
%\def\What{, Your degrees, etc.}
%
% Set the name of your Department or School here
% I honestly don't know why the negative spacing is necessary to get the alignment of the first line correct.  This must be a ``\tikz'' thing.
%%%%%%%%%%%%%%%%%%  School or Department %%%%%%%%%%%%%%%
\def\Where{\hspace{-1.2mm}\textbf{\color{osuorange}
School of Mechanical, Industrial, and Manufacturing Engineering
}}
%%%%%%%%%%%%  Additional Contact Information %%%%%%%%%%%
%
% Set your preferred primary contact address here.
\def\Address{320 Rogers Hall\\
2000 SW Monroe Ave
}
%
\def\CityZip{Corvallis, OR~~97331-6001}
%
% Set your OSU e-mail here
\def\Email{{\color{osuorange}email}: kyle.niemeyer@oregonstate.edu}
%
% Set your preferred OSU contact number here
\def\TEL{{\color{osuorange}P}: 541-737-5614}
%
% Set your department or personal website here
\def\URL{{\color{osuorange}URL}: \url{https://niemeyer-research-group.github.io}}
%
%%%%%%%%%%%%%%%%  Footer information  %%%%%%%%%%%%%%%%%%
%
%  The next line is for your college, used as a footer.  If you prefer not to have this, just comment out these lines in favor of the line labeled "[[Alternate]]" below
\def\school{\small{
  Oregon State University $\cdot$
     ~College of Engineering $\cdot$
     ~101 Covell Hall $\cdot$
     ~Corvallis, OR 97331-2409} }
% \def\school{~}  % [[Alternate]]
%
%%%%%%%%%%%%%%%%%%%%%  Signature line  %%%%%%%%%%%%%%%%%%%%%
%
% Set your signature line here.
% One can add a signature image in a PDF file using the following code; this requires a file called "signature_block.pdf" to be installed in the same folder as the .tex file.
% The vertical spacing (\vspace) and the scaling will have to be adjusted to get things to look correct for your particular signature image.
% Alternatively, comment out the following line in favor of the one labeled "[[Alternate]]" if you want to sign a paper copy of the letter.
%
%\signature{
%\vspace{-10mm}\includegraphics[scale=0.25]{signature_block.pdf}\\\vspace{-2mm}
%\name}
\signature{\name}  % [[Alternate]]

% This block sets up the address on the right-hand side of the header.
%
% The following lines just compile the information you set up into the LaTex letter variable "address" for later use.
%
%The following command "clears out" the default address so that it can be better set using \tikz
\address{}

\def\newaddress{
\Where\\
\Address\\
\CityZip\\
\TEL\\
\Email\\
\URL
}
%
%%%%%%%%%%%  DATE  %%%%%%%%%%%%%%%%%%%%%%%%%
% If you want a date different from the current date, comment out the next line in favor of the line labeled "[[Alternate]]".  The ``\vspace{10mm}'' just moves the date down a tiny bit so it doesn't interfere with the header.  This can be adjusted to your preference.
%
\date{\vspace{10mm} \today}
%\date{\vspace{10mm} 20 September 2020}  %[[Alternate]]
%
%%%%%%%%%%% Set the subject here if there is one  %%%%
\subject{KIN-17-0103.R1} % optional subject line

\begin{document}
%
%
%%%%%%%%  The "To" address goes here.
%
\begin{letter}{
               Prof.~Kevin Van Geem\\
               Associate Editor\\
               International Journal of Chemical Kinetics
               }
% This line sets up the return address to the right-side of the OSU logo.  The location is set with absolute node addresses using ``\tikz''.  It can still be a bit fussy, and you may need to alter this a little to get things to look right.  The bit that changes the position are the numbers in parentheses ``at (14.2,2.7)''
%
\begin{tikzpicture}[remember picture,overlay,,every node/.style={anchor=center}]
\node[text width=7cm] at (page cs:0.5,0.73){\small \newaddress};
\end{tikzpicture}

%%%%%%  The ``opening'' is just the methd of address you would like to use at the start of the letter.
%
\opening{Dear Professor Van Geem,}

%%%%%%%%%% Body of letter   %%%%%%%%%%%%%%
% The ``\lipsum[1-8]" command just fills the letter with 8 paragraphs of Latin for the purposes of filler.  Unless you really want to send filler Latin to someone, you will replace this command with actual text.  Do that here:
%
We enclose revisions for our manuscript ``ChemKED: a human- and machine-readable
data standard for chemical kinetics experiments``
and address the reviewers' comments in the following remarks.
We have indicated our additions and deletions using \addone{teal text}
for Reviewer \#1 and \addtwo{red text} for Reviewer \#2.


{\bf Reviewer \#1}
\begin{quoting}
Page 5, line 36: The following information `os' required ...
\end{quoting}

We have corrected the text on page 5: ``The following information
\deleteone{os}~\addone{is} required in each element of the sequence:''

\vspace{2em}


{\bf Reviewer \#2}
\begin{quoting}
    The authors properly answered the questions and comments. I suggest the publication
    of the manuscript after further minor modifications.

    Minor comments:

    p. 2, last lines:
    Notes: (1) the specification is available without registration; (2) I do not think
    that the requirement of an institutional e-mail address prevents the access of the public.
    ORIGINAL:
    Moreover, the standard remains inaccessible to the public, requiring potential
    users to register with the standard's authors using a valid institutional email
    address to access the specification and the growing database. \\
    SUGGESTED:
    Moreover, the potential users are required to register using a valid institutional
    email address to access the growing database.
\end{quoting}

We have modified the text on page 2 following the suggestion:
``Moreover, \deletetwo{the standard remains inaccessible to the public, requiring potential users
to register with the standard's authors using a valid institutional email address
to access the specification and the growing database}
\addtwo{potential users must register using a valid institutional email address
to access the growing database}.''

\begin{quoting}
    p. 3, first para:
    ORIGINAL:
    ... improved models for hydrogen/syngas [13], methanol/formaldehyde [14], and
    ethanol [15]. In all five cases, \\
    SUGGESTED:
    improved models for hydrogen [Proc. Combust. Inst.,  35, 589-596 (2015)],
    hydrogen/syngas [13], methanol/formaldehyde [14], and ethanol [15]. In all six cases,
\end{quoting}

We have adopted the suggested text on page 3:
``\ldots improved models for \addtwo{hydrogen [13]},
hydrogen\slash syngas [14], methanol\slash formaldehyde [15],
and ethanol [16]. In all \deletetwo{five}~\addtwo{six} cases \ldots''


\begin{quoting}
    p. 3, first para:
    ORIGINAL:
    contains 93,240 experimental data points in 1376 ReSpecTh XML files\\
    SUGGESTED:
    contains 93,326 data points in 1396 ReSpecTh XML files
\end{quoting}

We have adopted the suggested text on page 3:
``Their database currently contains \deletetwo{\num{93240}}~\addtwo{\num{93326}}
experimental data points in \deletetwo{1376}~\addtwo{1396}
ReSpecTh XML files [17].''

\begin{quoting}
    p. 7, last lines
    ``the 0-based index of the column'' (twice)
    I do not understand this term and maybe some other readers also will not.
\end{quoting}

We have clarified the description in both places on page 7:
``the \deletetwo{0-based} \addtwo{zero-based} index of the column storing the
volume information in the \yabox|values| array \addtwo{(e.g., 0 or 1)}''.

\begin{quoting}
    ``A sequence of sequences describing the values''
    I do not understand this term and maybe some other readers also will not.
\end{quoting}

We have clarified this description on page 8:
``A \deletetwo{sequence of sequences describing the values of
the volume at the time points.}\addtwo{sequence of time-volume pairs describing the values
of the volume at different times}''.

\begin{quoting}
    p. 9
    ``If the volume-history of an RCM experiment is provided in the
    ChemKED file, it is stored in this attribute as a namedtuple ,''
    Question: Is this format able to handle the case when a shock tube measurement
    is accompanied with a pressure history?
\end{quoting}

The current version of ChemKED can represent a constant preignition pressure rise
in shock tubes through the \yabox|pressure-rise| key. It does not currently
support specifying a pressure time-history, although we are planning to extend
support for more time histories for both shock tubes and rapid compression
machines in a future version (for, e.g., pressure, volume, piston position, light
emission, OH emission, and absorption). See \url{https://github.com/pr-omethe-us/PyKED/issues/91}
for our discussion on these plans.

We added a footnote on page 7 describing these plans:
``\addtwo{Future versions of ChemKED will support specifying additional time histories for
rapid compression machine and shock tube experiments (e.g., pressure, volume, light emission, OH emission).
These will follow the specification format given for volume.}''

\begin{quoting}
    p. 10, first para:
    ORIGINAL:
    not all ReSpecTh files may be converted to ChemKED files
    SUGGESTED:
    not all ReSpecTh files may be equivalently converted to ChemKED files
\end{quoting}

We have adopted the suggested text on page 10:
``not all ReSpecTh files may be \addtwo{equivalently} converted to ChemKED files''.

\begin{quoting}
    Reference [21]:
    ``döt Net''   ???????
\end{quoting}

This is the correct surname for one of the creators of the YAML data format: Ingy döt Net.

\vspace{1em}

%%%%%%% ``closing'' sets the sign-off line.
\closing{Sincerely,}

% Comment out/in the lines below as necessary
%\encl{If an enclosure is provided, let them know what it is.}

%\ps{A postscript if that is a thing you do.}

%\cc{Someone Who Cares (and is copied).}

\end{letter}

\end{document}
