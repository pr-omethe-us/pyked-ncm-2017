\documentclass[a4paper,10pt]{elsarticle}
\usepackage[utf8]{inputenc}

\biboptions{sort&compress, square, comma}
\usepackage[left=1in, right=1in, top=1in, bottom=1in]{geometry}

\usepackage[usenames, dvipsnames]{xcolor}
\usepackage{siunitx}
\sisetup{group-separator={,},
     detect-all,
     binary-units,
     list-units = single,
     range-units = single,
     range-phrase = --,
     per-mode = symbol-or-fraction,
     separate-uncertainty = true,
     multi-part-units = single,
     list-final-separator = {, and }
%    scientific-notation = fixed
}
\providecommand{\hly}[1]{\colorbox{yellow}{#1}}
\providecommand{\hlb}[1]{\colorbox{SkyBlue}{#1}}
\providecommand{\hlg}[1]{\colorbox{green}{#1}}

% bump up section font size
\usepackage{sectsty}
\sectionfont{\fontsize{12}{15}\selectfont}

\usepackage[colorlinks=true]{hyperref}
\usepackage{todonotes}

\setlength{\parindent}{0in}
\setlength{\parskip}{0.4cm}

\newenvironment{reviewer}{\vspace{0.5\baselineskip}\begingroup\itshape\textbf{Reviewer:}}{\endgroup\vspace{0.5\baselineskip}}
\newenvironment{response}{\vspace{0.5\baselineskip}\textbf{Our Response:}}{\vspace{0.5\baselineskip}}
\newcommand{\review}[1]{{\itshape#1}}
\newcommand{\respond}[1]{\textbf{Our Response:} #1}

\begin{document}

We thank the reviewers for their helpful and detailed comments, and have addressed their remarks here.
The manuscript has been significantly revised, and we marked significant changes corresponding to suggestions from Reviewers \#1 and \#2 in \hlg{yellow} and \hly{green}, respectively.


\section*{Reviewer \#1:}

\begin{reviewer}
    %
    Currently four database formats are offered for storing combustion experimental data. These are
    CloudFlame (Excel/CSV format), PrIMe (XML-based format), ReSpecTh (also XML-based format), and
    ChemKED (YAML-based format). Elaboration of the last format is reported in the submitted
    manuscript. The motivation for the elaboration of this new format was based on the statement
    that YAML is more human readable than XML. I think that this is simply not true. Even a
    first-time user can read the raw datafiles of both formats with equal efficiency. Also,
    currently the suggested new format is much more limited (e.g. it is able to encode ignition
    delay time experiments only). The presently planned features of ChemKED are already available in
    the other three formats.

    However, I suggest the publication of an improved and shortened version of the manuscript in
    IJCK. The reason is that a permanent and laboratory-independent storage of experimental data is
    an important issue and I agree that a public discussion promotes the acceptance of a widely used
    common format.
    %
\end{reviewer}

\begin{response}
    %
    We would like to thank the reviewer for their comments. Detailed responses to each comment are
    below.
    %
\end{response}

\begin{reviewer}
    %
    I think that the paper contains too much information science details. Still, as the authors
    note, the description is not comprehensive and they refer to the user manual. I suggest the
    creation of a significantly abridged version that contains the basic information technology
    points only and does not discuss the details. For example, Chapter 4 could be deleted and
    Chapters 2 and 3 may also be shortened.
    %
\end{reviewer}

\begin{response}
    %
    Our intention is to describe the system to an audience that may not be familiar with the
    technical details of how the system is implemented. Thus, we feel that the length of the
    manuscript is a necessary outcome of providing the background for this information. However, the
    reviewer is correct that it is not ideal to leave much of the information in the documentation
    only. In combination with the suggestions of Reviewer 2, we have expanded the paper to provide
    more detail where necessary.
    %
\end{response}

\begin{reviewer}
    %
    p.~1 ``Competing standards such as the Cantera [2] CTI format or FlameMaster [3] lag behind
    considerably, although use of the former continually grows due to Cantera's open availability.''

    Note that FlameMaster is also freely available from the GitLab repository of the developers.
    %
\end{reviewer}

\begin{response}
    We attempted to access the GitLab repository at \url{https://git.rwth-aachen.de/ITV/FlameMaster}, but
    it is a closed repository. Thus, while it may be freely available to those with access, it is not
    \emph{openly} available.
\end{response}

\begin{reviewer}
    %
    p.~2 ``Second, the XML format is intended to be a machine-readable markup language rather than a
    data format, and its lack of human readibility presents a barrier to creating and working with
    database files.''

    XML is a text format and it is fully human readable, similarly to YMAL. Even a
    first-time user is able to read the chemical content of a  PrIMe XML file and an experienced
    user is able to write such a file.
    %
\end{reviewer}

\begin{response}
    %
    We certainly recognize that XML is typically written in ASCII or UTF-8 (plain text) files, which
    a human can hypothetically decipher. Our argument is that XML is significantly harder for a
    human to read and write than YAML. This is primarily because XML has a lower signal-to-noise
    ratio than YAML; it requires many more characters to express a data structure in XML than in
    YAML. For instance, XML requires opening and closing tags, with the associated angle brackets,
    for every field while YAML does not; in XML, information can be specified as properties of tags
    or as elements (requiring the user to know where to look for information) while YAML has only 3 basic data structures (keys, values, and lists).

    We note that although this verbosity in XML is a disadvantage when humans attempt to read or
    write such files, it allows much greater flexibility in specifying how a document should be
    marked up and provides XML with many advantages in other areas. However, since YAML is intended
    as a simple data format, and has a much simpler overall format, we argue that it is therefore
    much easier for humans to work with.

    We have modified the text to clarify this point, including describing the lack of human
    readability as a formatting issue (but one that is inherent to XML), rather than a technical
    issue.
    %
\end{response}

\begin{reviewer}
    %
    I suggest adding to following block to this para:

    ``The original PrIMe was on an Amazon cloud server. Recently it was shifted to the computer
    center of DLR Stuttgart. At the new location it is called ``PrIMe 2.0''
    [\url{https://prime.cki-know.org/}]. This is identical to the original PrIMe apart from
    technical changes due to the different computation background. Professor Frenklach plans to
    develop further PrIMe (using name PrIMe 3.0) and this version will be availabe at
    [\url{http://primekinetics.org/}].''
    %
\end{reviewer}

\begin{response}
    %
    We do not think that the technical details of the implementation of the original PrIMe system
    are relevant to this work, so we have not included them. In addition, we cannot know
    the plans of Prof. Frenklach, and we do not feel it would be appropriate to comment on them in
    our manuscript. As such, we have not added these suggestions to the paper. We have, however,
    added the new URL to the current manuscript, and thank the reviewer for pointing us to that.
    %
\end{response}

\begin{reviewer}
    %
    ``For example, where PrIMe uses internal bibliographic references, ReSpecTh adds a field for
    typical bibliographic data.''

    I suggest the following extension:

    ``For example, where PrIMe uses internal bibliographic references, ReSpecTh adds a field for
    typical bibliographic data, including the DOI of the cited arecicle. Each datafile has a unique
    own DOI and unique URL address.
    [\url{http://respecth.chem.elte.hu/respecth/reac/CombustionData.php}]''. Also species can be
    identified by InCHI or SMILES.
    %
\end{reviewer}

\begin{response}
    %
    We have incorporated these modifications into the suggested paragraph, although we have modified
    the format slightly to better fit the flow of the text.
    %
\end{response}

\begin{reviewer}
    %
    ``The ReSpecTh standard also provides machine-readable formats for describing ignition
    experiments, including a field for the definition of autoignition and the ability to specify
    facility-specific effects.''

    Modified text:

    ``The ReSpecTh standard provides formats that are both human and machine-readable for describing
    ignition experiments, including a field for the definition of autoignition and the ability to
    specify facility-specific effects. Also, it describes PSR and flow reactor speciation data,
    burner stabilized flame speciation data and burning velocity data.''
    %
\end{reviewer}

\begin{response}
    %
    We have incorporated most of the recommended text in our revision. However, we did not add
    ``human-readable'', consistent with the definition used throughout (i.e., the XML file is more
    verbose and less easily understood).
    %
\end{response}

\begin{reviewer}
    %
    ``However, ReSpecTh experimental files are another XML-based format, and, as such, suffer from
    the same usability issues as PrIMe.''

    I think that the limits of an XML database is set by the actual set of tags and not the format.
    %
\end{reviewer}

\begin{response}
    %
    By ``usability'' of a format, we mean whether files are easily usable and understandable by
    people, rather than the technical limitations of a format. We have changed the word
    ``usability'' to ``readability'' to emphasize this.
    %
\end{response}

\begin{reviewer}
    %
    ``Moreover, the standard remains closed, requiring potential users to register with the
    standard's authors to access the specification and the growing database.''

    I suggest instead:

    ``Registration is open for everyone who provides a valid institutional e-mail address.''
    %
\end{reviewer}

\begin{response}
    %
    We have updated the text to reflect this information; however, we argue that requiring
    registration to view the specification and data limits the possibility of use of the standard
    and differs from the openness that we are advocating, where the specification and data files are
    open for anyone to view, with or without registration.
    %
\end{response}

\begin{reviewer}
    %
    ``... hydrogen/syngas [11] and ethanol [12]. In all four cases, they converted numerous
    experimental datasets from the literature into the ReSpecTh standard.''

    Extended version:

    ``... hydrogen/syngas [11], methanol and formaldehyde [C. Olm, T. Varga, E. Valko, H. J. Curran,
    T. Turanyi: Uncertainty quantification of a newly optimized methanol and formaldehyde combustion
    mechanism, Combustion and Flame, in press] and ethanol [12]. In all five cases, they converted
    numerous experimental datasets from the literature into the ReSpecTh standard. The database
    currently contains  93240 experimental data points in 1376 XML-format data files
    [\url{http://respecth.chem.elte.hu/respecth/reac/CombustionData.php}].''
    %
\end{reviewer}

\begin{response}
    We added the suggested text and reference.
\end{response}

\begin{reviewer}
    %
    ``type (required, string): How ignition delay was measured;''

    The options described here is a very limited subset of the ignition delay time definitions used
    in experimental articles. For example, backward or forward interpolated values of a measured
    quantity are further possibilities.
    %
\end{reviewer}

\begin{response}
    %
    We have implemented an option for baseline extrapolation of the slope. We have added the
    description of this ignition type to the mentioned portion of the manuscript. To our knowledge,
    the options now supported represent all of the common methods of specifying the ignition delay
    in rapid compression machine and shock tube experiments.
    %
\end{response}

\begin{reviewer}
    %
    p.~8 ``In the latter case, some information may be lost, as ChemKED files both encode more
    details and specify more granularity of some information than ReSpecTh files.''

    This was true for ReSpecTh format 1.0, but no longer true for ReSpecTh Kinetic Dataformat 2.0.
    The latter covers all info of the ChemKED files (and more).
    %
\end{reviewer}

\begin{response}
    %
    This is still the case, even in ReSpecTh format 2.0. For instance, ChemKED files support
    specifying the initial composition as a mass fraction, which ReSpecTh does not support. We have
    modified the sentence to indicate that there are some incompatibilities in both directions, not
    all ChemKED files can be converted to ReSpecTh files, and not all ReSpecTh files can be
    converted to ChemKED files due to missing data elements in both directions.
    %
\end{response}

\begin{reviewer}
    %
    Missing data in the references: Year is missing is refs.~[5], [6], [9], [10], [11], [12],
    [21],[25],[27].

    [3] Please add reference: \url{https://www.itv.rwth-aachen.de/index.php?id=13}  Also, the
    latest version is from 2017.

    [7] Further info:  Paper P1-04, ISBN 978-963-12-1257-0

    [8] Please refer to ReSpecTh Kinetics Data Format Specification 2.0 instead.
    %
\end{reviewer}

\begin{response}
    %
    We corrected those references; thank you for pointing out the missing\slash old information.
    %
\end{response}

%%%%%%%%%%%%%%%%%%%%%%%%%%%%%%%%%%%%%%%%%%%%%%%
\section*{Reviewer \#2:}

\begin{reviewer}
    %
    My main concern with the creation of a publically available database, formatted with ChemKED, is
    that it seems that the creation of the files has to be done manually, i.e. susceptible for
    errors. This is especially true if the experiment comprises a big set of datapoints and/or
    hundreds of measured species. Furthermore, the YAML data serialization format is indeed
    human-readable, but it does not provide an easy summary/overview of the experimental data, such
    as an excel or CSV file would give, and therefore, people might be not willing to put the extra
    effort.
    %
\end{reviewer}

\begin{response}
    %
    We agree with the reviewer, insofar as manual creation of the files may lead to errors in the
    encoded data. However, this will be the case wherever manual processing is required, regardless
    of the database format, and could even be the case for tables or figures published in journal
    articles. Thus, we do not feel that manual file creation is an inherent disadvantage of our
    format. Moreover, as we discuss below, we are working with experimental teams to integrate the
    creation of ChemKED files into their workflows to try to mitigate some of these problems.

    On the second point, we agree with the reviewer that CSV or Excel files may give an adequate
    summary or overview of the data, but only if the files are opened with some appropriate
    software, e.g., Microsoft Excel (particularly for Excel-formatted files). Thus, in this sense,
    CSV or Excel files do not offer any advantage over ChemKED files (or any other database files)
    that must also be processed with an appropriate software package (PyKED, in the case of ChemKED
    files) so that a summary can be produced. Furthermore, in the specific case of PyKED/ChemKED,
    it is trivial to produce an Excel or CSV file so that one may view such a summary if desired.
    See also the response to a similar comment below.
    %
\end{response}

\begin{reviewer}
    %
    1.~How are the authors planning on `forcing' the experimentalists to publish their data in this
    new ChemKED file format?
    %
\end{reviewer}

\begin{response}
    %
    While we cannot force anyone to use a particular format, we hope that by defining a format and
    ecosystem with desirable features---including inherent openness and usability---experimentalists
    will begin adopting a community-standard format. In fact, we have already received expressions
    of interest from various experimental groups, and hope this will grow as we continue to improve
    and add features. We also hope that at a certain point, with wide adoption, sharing experimental
    data in a standard format will become the expected norm (e.g., when submitting an article for
    publication) like sharing a Chemkin-format chemical kinetic model.

    Furthermore, we are willing to work (and are currently working) directly with experimentalists
    to provide them easy-to-use conversion scripts from internal data formats. In return, they are
    helping us define what data and information needs to go in the specification for their
    experiment.
    %
\end{response}

\begin{reviewer}
    %
    2.~Am I correct to state that ChemKED is a YAML file with a predefined structure?
    %
\end{reviewer}

\begin{response}
    %
    Yes, ChemKED files are files in YAML format, with a prescribed structure and set of contents.
    The required structure and contents are defined by our schema, which is documented in this
    manuscript and in the online documentation.
    %
\end{response}

\begin{reviewer}
    %
    3.~I think the article would be more appealing if it would have at least one figure as an
    example.
    %
\end{reviewer}

\begin{response}

\end{response}

\subsection*{Abstract and introduction}

\begin{reviewer}
    %
    4. In the abstract, it is mentioned that the Python-based package PyKED creates ChemKED-
    formatted files. However in the article there is no evidence that PyKED aides in creating
    ChemKED-formatted files.
    %
\end{reviewer}

\begin{response}
    %
    The reviewer is correct, and this was a typo in the abstract. We have clarified that PyKED is
    intended as software to validate and work with ChemKED files.

    That said, we do have ongoing work to create a graphical user interface to create ChemKED files,
    as this was requested by several of the experimental teams that have expressed interest in the
    ChemKED format. The work is available publicly at
    \url{https://github.com/pr-omethe-us/ChemKED-gui}.
    %
\end{response}

\begin{reviewer}
    %
    5.~Page 1, line 31: The future development plans include support for jet stirred reactors,
    however, this is not mentioned in section 5, Conclusions and future work, nor is there any
    reference to it on the GitHub wiki roadmap of PyKED.
    %
\end{reviewer}

\begin{response}
    %
    Thank you for the comment, we have corrected this oversight in the roadmap and the conclusions.
    %
\end{response}

\begin{reviewer}
    %
    6. I don't completely agree with the second part of the statement on page 2 line 16-18:
    ``Second, the XML format is intended to be a machine-readable mark-up language rather than a
    data format, and its lack of human readability presents a barrier to creating and working with
    database files.''

    In my opinion the YAML format resembles the XML format strongly. XML format is human-readable,
    however, because of the html-tags and \texttt{<>} characters, the YAML format might be easier to
    read. If XML would lack human readability as it is stated in the article, it would not be such a
    popular format, and I don't think PrIMe and ReSpecTh would be using it.
    %
\end{reviewer}

\begin{response}
    %
    We agree with the reviewer that XML is technically human readable by an expert (as opposed to,
    say, a binary format), and also that YAML is easier to read---by ``human-readable'', we really
    mean human-\emph{understandable}. We have modified the text to clarify this point; please see
    also the response to a similar comment from Reviewer \#1.
    %
\end{response}

\begin{reviewer}
    %
    7.~An initial version of the ChemKED format was introduced by Niemeyer, as is mentioned in the
    introduction. The mentioned article describes a Python-package called PyTeCK. How does this
    PyTeCK package relate to the PyKED package that is presented in the submitted article?
    %
\end{reviewer}

\begin{response}

\end{response}

\begin{reviewer}
    %
    8.~Page 3, line 10: ``Our motivations are similar to those of '', `to' is missing.
    %
\end{reviewer}

\begin{response}
    %
    We have edited that sentence to correct the error.
    %
\end{response}

\subsection*{Overview of ChemKED format}

\begin{reviewer}
    %
    9.~A reference should not be a required mapping of an experimental dataset. If the intention is
    to make the community to adopt the ChemKED file structure, it might be easier to do so if
    experimentalists can use the structure internally as well, i.e. to read, store, and share
    non-published internal data.
    %
\end{reviewer}

\begin{response}
    %
    We agree with the reviewer that some of the choices currently enforced in the PyKED package and
    the ChemKED schema do not lend themselves to easy informal use. We presently recommend that
    users who wish to use ChemKED files in this way include an incomplete reference field since the
    only required subfields of the reference field are the authors and the year.

    In addition, there is a \verb|skip_validation| argument to the \verb|ChemKED| class constructor.
    This option turns off all of the validation, including the validation of the reference material
    and enforcement of required fields. We recognize that this is a limitation at present, if users
    wish to, for example, validate that their units are correct but do not wish to validate some
    other aspect of the file. Work is ongoing to improve this situation, and discussion is contained
    in a public issue on GitHub: \url{https://github.com/pr-omethe-us/PyKED/issues/39}
    %
\end{response}

\begin{reviewer}
    %
    10.~Page 4, line 32: ``The second section of the file encodes the experimental data''. Perhaps
    it is easier to follow if this is briefly mentioned before the authors explain the details of
    the first section of the ChemKED format.
    %
\end{reviewer}

\begin{response}

\end{response}

\begin{reviewer}
    %
    11.~Why do both \texttt{atomic-composition} and \texttt{elemental-composition} exist as
    synonyms? Wouldn't it be clearer if only one of the two can be used as a valid input?
    %
\end{reviewer}

\begin{response}

\end{response}

\begin{reviewer}
    %
    12.~Is it possible to create for example two \texttt{common-properties} \texttt{composition}
    blocks A and B and refer to them in the \texttt{DataPoint} keys?
    %
\end{reviewer}

\begin{response}

\end{response}

\begin{reviewer}
    %
    13.~Page 6, line 12 describes the initial efforts of building an open repository of ChemKED
    files, on GitHub. Is this the most efficient way for making a publically available database? One
    issue with this is the fact that adding files is done via pull requests, which have to be
    verified and approved by the owners of the repository. Therefore a certain delay is inevitable.
    %
\end{reviewer}

\begin{response}
    %
    We agree that submitting files to the ChemKED database on GitHub will inevitably incur some
    delay, however we see the human intervention step as necessary and important. In much the same
    way that review of manuscripts provides a certain surety of the minimal quality of the article,
    we see review of data being added to the database as a minimal quality check; for instance, does
    the reference exist, has the data already been added to the database, are the files styled
    consistently, etc. Furthermore, such human review can help to catch some errors that may be
    introduced by manual creation of the files, as pointed out by the reviewer previously. Finally,
    we would note that all of the existing database have a similar requirement, where users submit
    data files for inclusion that are approved by the database owners after some review. In this
    sense, we actually see the Pull Request mechanism of GitHub as an advantage, since all reviews
    can be done publicly and some of the review procedures (style checks, reference validity, etc)
    may be done automatically by a script associated with the GitHub repository. In this way, the
    entire community can contribute to improving the quality of the database.
    %
\end{response}

\begin{reviewer}
    %
    14.~Page 6, line 18 states that a converter exists from ReSpecTh to ChemKED. But isn't this
    feature added when going from v0.1.6 to v0.2.0, which makes page 3 line 21: v0.1.6 not correct.
    %
\end{reviewer}

\begin{response}
    %
    In principle, the version of the ChemKED database format and the version of the PyKED Python
    package are entirely independent. Thus, since adding the converters incurred no changes to the
    database format, the description of the format in the manuscript is a description of v0.1.6,
    which happens to be identical to v0.2.0. However, in practice, the versions of PyKED and the
    ChemKED database have been coupled up to now, and we have updated the version mentioned in the
    text to vX.X.X, which we released in conjunction with submitting this response after the helpful
    comments of the reviewers.
    %
\end{response}

\begin{reviewer}
    %
    15.~Page 6, line 21, is the `,' necessary after fork?
    %
\end{reviewer}

\begin{response}
    %
    We removed the extraneous comma.
    %
\end{response}

\subsection*{PyKED architecture}

\begin{reviewer}
    %
    16.~The Python package Cerberus is used to validate the format and content of the given ChemKED
    file. How does this package handles/reports the inconsistencies with the schemas? As far as my
    local test went with giving a corrupted YAML file to the ChemKED initiator, the package just
    spits out a Python error, which might be not the most user-friendly solution.
    %
\end{reviewer}

\begin{response}

\end{response}

\begin{reviewer}
    %
    17.~A recurring issue with actively developing codes is backwards compatibility. How are the
    authors planning on addressing this issue?
    %
\end{reviewer}

\begin{response}

\end{response}

\begin{reviewer}
    %
    18.~Page 7, line 12, \texttt{mole-percent}, \texttt{mole-fraction}, and \texttt{mass-fraction}
    include a dash in their name, whereas this is not the case for the example on page 5 line 48 and
    in the description of the composition element on page 4, line 47.

    19.~Same issue as \#18 holds for page 7, line 17.
    %
\end{reviewer}

\begin{response}

\end{response}

\begin{reviewer}
    %
    20.~I think it would be more clear if \texttt{volume\_history}, \texttt{compression\_time},
    \texttt{compressed\_temperature}, and \texttt{compressed\_pressure} would be combined.

    The line: `if the ChemKED file encodes an RCM experiment' is unnecessarily repeated.
    %
\end{reviewer}

\begin{response}

\end{response}

\begin{reviewer}
    %
    21.~Page 7, line 39: ``Future versions of PyKED may support this feature via, e.g. online
    lookup''. There are two main caveats with online lookups: a) It will make the code slower if the
    internet connection is not strong, b) an internet connection is necessary for the code to be
    executed.
    %
\end{reviewer}

\begin{response}

\end{response}

\begin{reviewer}
    %
    22.~Page 7, line 40: Why is it interesting to create the YAML file again with the
    \texttt{write\_file()} function. Creating a summary excel/CSV file would be more useful.
    %
\end{reviewer}

\begin{response}

\end{response}

\subsection*{Usage examples}
\subsubsection*{RCM modelling with varying reactor volume}

\begin{reviewer}
    %
    23.~Page 10, line 29--30: ``A user might want to load the information'' Avoid the term `might'.
    It's one of the main goals of using ChemKED and its associated Python-package PyKED to load the
    information necessary for simulating the experiment.
    %
\end{reviewer}

\begin{response}

\end{response}

\begin{reviewer}
    %
    24.~The Python code snippets are not always initiated in the text.
    %
\end{reviewer}

\begin{response}

\end{response}

\begin{reviewer}
    %
    25.~Page 10, line 55: add more commentary text. It is not clear that those lines create the
    necessary instances for Cantera, rather than ChemKED.
    %
\end{reviewer}

\begin{response}

\end{response}

\begin{reviewer}
    %
    26.~Page 11, line 7: Where is this `air.xml' file coming from. This is not clear from the text.
    %
\end{reviewer}

\begin{response}

\end{response}

\subsubsection*{Shock tube modelling with constant volume}

\begin{reviewer}
    %
    27.~The table overlapping page 11 and 12 can be reduced by truncating the \texttt{DataPoints},
    similarly as is done with the table starting at page 9, line 31 and ending at page 10, line 27.
    %
\end{reviewer}

\begin{response}

\end{response}

\begin{reviewer}
    %
    28. Python code snippet on page 13: (a.) Add more commentary text. (b.) Changing the example to
    enable multithreading might make it more appealing to start using ChemKED as a data format in
    addition with Cantera as a simulation tool.
    %
\end{reviewer}

\begin{response}

\end{response}

\subsection*{Conclusions and future work}

\begin{reviewer}
    %
    29.~No conclusion is mentioned for the usage examples.
    %
\end{reviewer}

\begin{response}

\end{response}

\section*{References}
\bibliography{rebuttal}
\bibliographystyle{elsarticle-num}

\end{document}
