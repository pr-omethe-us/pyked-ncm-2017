\documentclass[a4paper,10pt]{elsarticle}
\usepackage[utf8]{inputenc}

\biboptions{sort&compress, square, comma}
\usepackage[left=1in, right=1in, top=1in, bottom=1in]{geometry}

\usepackage[usenames, dvipsnames]{xcolor}
\usepackage{siunitx}
\sisetup{group-separator={,},
     detect-all,
     binary-units,
     list-units = single,
     range-units = single,
     range-phrase = --,
     per-mode = symbol-or-fraction,
     separate-uncertainty = true,
     multi-part-units = single,
     list-final-separator = {, and }
%    scientific-notation = fixed
}
\providecommand{\hly}[1]{\colorbox{yellow}{#1}}
\providecommand{\hlb}[1]{\colorbox{SkyBlue}{#1}}
\providecommand{\hlg}[1]{\colorbox{green}{#1}}

% bump up section font size
\usepackage{sectsty}
\sectionfont{\fontsize{12}{15}\selectfont}

\usepackage[colorlinks=true]{hyperref}
\usepackage{todonotes}

\setlength{\parindent}{0in}
\setlength{\parskip}{0.4cm}

\begin{document}

We thank the reviewers for their helpful and detailed comments, and have addressed their remarks here.
The manuscript has been significantly revised, and we marked significant changes corresponding to suggestions from Reviewers \#1 and \#2 in \hlg{yellow} and \hly{green}, respectively.


\section*{Reviewer \#1:}

\textit{Currently four database formats are offered for storing combustion experimental data. These are CloudFlame (Excel/CSV format), PrIMe (XML-based format), ReSpecTh (also XML-based format), and ChemKED (YAML-based format). Elaboration of the last format is reported in the submitted manuscript. The motivation for the elaboration of this new format was based on the statement that YAML is more human readable than XML. I think that this is simply not true. Even a first-time user can read the raw datafiles of both formats with equal efficiency. Also, currently the suggested new format is much more limited (e.g. it is able to encode ignition delay time experiments only). The presently planned features of ChemKED are already available in the other three formats.}

\textit{However, I suggest the publication of an improved and shortened version of the manuscript in IJCK. The reason is that a permanent and laboratory-independent storage of experimental data is an important issue and I agree that a public discussion promotes the acceptance of a widely used common format.}

\textbf{Reviewer comments:}
\textit{I think that the paper contains too much information science details. Still, as the authors note, the description is not comprehensive and they refer to the user manual. I suggest the creation of a significantly abridged version that contains the basic information technology points only and does not discuss the details. For example, Chapter 4 could be deleted and Chapters 2 and 3 may also be shortened.}

\textbf{Our response:}



\textbf{Reviewer comment:}
\textit{p.~1 ``Competing standards such as the Cantera [2] CTI format or FlameMaster [3] lag behind considerably, although use of the former continually grows due to Cantera?s open availability.''
Note that FlameMaster is also freely available from the GitLab repository of the developers.}

\textbf{Our response:}



\textbf{Reviewer comment:}
\textit{p.~2 ``Second, the XML format is intended to be a machine-readable markup language rather than a data format, and its lack of human readibility presents a barrier to creating and working with database files.''
XML is a text format and it is fully human readable, similarly to YMAL. Even a first-time user is able to read the chemical content of a  PrIMe XML file and an experienced user is able to write such a file.}

\textit{I suggest adding to following block to this para: ``The original PrIMe was on an Amazon cloud server. Recently it was shifted to the computer center of DLR Stuttgart. At the new location it is called ``PrIMe 2.0'' [\url{https://prime.cki-know.org/}]. This is identical to the original PrIMe apart from technical changes due to the different computation background. Professor Frenklach plans to develop further PrIMe (using name PrIMe 3.0) and this version will be availabe at [\url{http://primekinetics.org/}].''
}

\textbf{Our response:}



\textbf{Reviewer comment:}
\textit{``For example, where PrIMe uses internal bibliographic references, ReSpecTh adds a field for typical bibliographic data.'' 
I suggest the following extension:
``For example, where PrIMe uses internal bibliographic references, ReSpecTh adds a field for typical bibliographic data, including the DOI of the cited arecicle. Each datafile has a unique own DOI and unique URL address. [\url{http://respecth.chem.elte.hu/respecth/reac/CombustionData.php}]''. Also species can be identified by InCHI or SMILES.
}

\textbf{Our response:}


\textbf{Reviewer comment:}
\textit{``The ReSpecTh standard also provides machine-readable formats for describing ignition experiments, including a field for the definition of autoignition and the ability to specify facility-specific effects.''
Modified text:
``The ReSpecTh standard provides formats that are both human and machine-readable for describing ignition experiments, including a field for the definition of autoignition and the ability to specify facility-specific effects. Also, it describes PSR and flow reactor speciation data, burner stabilized flame speciation data and burning velocity data.''
}

\textbf{Our response:}



\textbf{Reviewer comment:}
\textit{``However, ReSpecTh experimental files are another XML-based format, and, as such, suffer from the same usability issues as PrIMe.''
I think that the limits of an XML database is set by the actual set of tags and not the format.
}

\textbf{Our response:}



\textbf{Reviewer comment:}
\textit{``Moreover, the standard remains closed, requiring potential users to register with the standard's authors to access the specification and the growing database.''
I suggest instead:
``Registration is open for everyone who provides a valid institutional e-mail address.''
}

\textbf{Our response:}


\textbf{Reviewer comment:}
\textit{
``... hydrogen/syngas [11] and ethanol [12]. In all four cases, they converted numerous experimental datasets from the literature into the ReSpecTh standard.''
Extended version:
``... hydrogen/syngas [11], methanol and formaldehyde [C. Olm, T. Varga, E. Valko, H. J. Curran, T. Turanyi: Uncertainty quantification of a newly optimized methanol and formaldehyde combustion mechanism, Combustion and Flame, in press] and ethanol [12]. In all five cases, they converted numerous experimental datasets from the literature into the ReSpecTh standard. The database currently contains  93240 experimental data points in 1376 XML-format data files [\url{http://respecth.chem.elte.hu/respecth/reac/CombustionData.php}]''.
}

\textbf{Our response:}



\textbf{Reviewer comment:}
\textit{``type (required, string): How ignition delay was measured;''
The options described here is a very limited subset of the ignition delay time definitions used in experimental articles. For example, backward or forward interpolated values of a measured quantity are further possibilities.  
}

\textbf{Our response:}


\textbf{Reviewer comment:}
\textit{p.~8 ``In the latter case, some information may be lost, as ChemKED files both encode more details and specify more granularity of some information than ReSpecTh files.''
This was true for ReSpecTh format 1.0, but no longer true for ReSpecTh Kinetic Dataformat 2.0. The latter covers all info of the ChemKED files (and more). 
}

\textbf{Our response:}


\textbf{Reviewer comment:}
\textit{Missing data in the references: Year is missing is refs. [5], [6], [9], [10], [11], [12], [21],[25],[27].
}

\textit{[3] Please add reference: \url{https://www.itv.rwth-aachen.de/index.php?id=13}
Also, the latest version is from 2017.
}

\textit{[7] Further info:  Paper P1-04, ISBN 978-963-12-1257-0}

\textit{[8] Please refer to ReSpecTh Kinetics Data Format Specification 2.0 instead.}

\textbf{Our response:}

%%%%%%%%%%%%%%%%%%%%%%%%%%%%%%%%%%%%%%%%%%%%%%%
\section*{Reviewer \#2:}



\section*{References}
\bibliography{rebuttal}
\bibliographystyle{elsarticle-num}

\end{document}
